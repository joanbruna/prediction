\section{Source Separation Algorithm}
\label{algosec}
We show in this section how the inverse problem of source separation 
can be solved via a sparse NMF in the pyramid scattering domain, followed by phase recovery.
We consider the supervised 
monoaural source separation problem (\ref{ssep}), 
 in which the components $x_i$, $i=1,2$ come from sources for 
 which we have training data $X_i=\{x_{ij}\}_{j \leq K}$, 
and one is asked to produce estimates $\widehat{x_i}$. 

\begin{table*}[t]
\centering
\begin{tabular}{l|c|c|c || c |c |c }
  \hline\hline
  & \multicolumn{3}{c||}{Speaker-Specific} & \multicolumn{3}{c}{Multi-Speaker} \\
  \hline
 & SDR & SIR & SAR & SDR & SIR & SAR\\
\hline
NMF  & 5.6 [1.8] & 13.4 [2.8] &   6.9 [1.3] & 6.1 [2.9] &   14.1 [3.8] & 7.4 [2.1] \\
\hline
\emph{scatt-NMF\textsubscript{1}} & 8.6 [1.7]  & {\bf 16.8} [3.3]  & 9.6 [1.4] &  6.2 [2.8] &   13.5 [3.5] & 7.8 [2.2] \\
\emph{scatt-NMF\textsubscript{2}} &  {\bf 8.9} [1.5] & {\bf 16.8} [2.4]  & {\bf 9.9} [1.3]  &  {\bf 6.9} [2.7] & {\bf 16.0} [3.5]  & {\bf 7.9} [2.2] \\
  \hline
  \hline
\end{tabular}
\caption{Separation with speakers-specific and multi-speaker settings. Average SDR, SIR and SAR (in $dB$) for NMF and proposed  and \emph{scatt-NMF\textsubscript{2}}. Standard deviation of each result shown between brackets. \label{ta:eval}}
%\vspace{-2ex}
\end{table*}

Let us consider for simplicity the features $\Phi_j(x_i)$, $j=1,2$, $i=1,2$, $x_i \in X_i$, obtained by localizing the scattering features of two different resolutions at their 
corresponding sampling rates. Therefore, $\Phi_1$ is equivalent to a CQT and carries more localized information than $\Phi_2$. 
On the other hand, $\Phi_2$ is stable at representing longer temporal contexts. We train independent models for each source at each resolution by solving,
\begin{equation}
\min_{D^i_j \geq 0} \, \sum_{x\in X_i}   \min_{z \geq 0}  \frac{1}{2}  \| \Phi_j(x) - D^j_i z \|^2 + \lambda_j \| z \|_1~.
\nonumber
\end{equation}
%\begin{equation}
%\min_{D_i\geq 0, Z_i\geq 0} \frac{1}{2} \| \Phi(X_i) - D_i \, Z_i \|_F^2 + \lambda \| Z_i \|_1~.
%\end{equation}
where $D^j_i$ represents the non-negative dictionary of source $i$ at resolution $j$. 
This model exploits the linearization properties of scattering coefficients since it 
searches low-dimensional linear approximations. 

At test time, given and input $x$, $x_1$ and $x_2$ are estimated by minimizing
\begin{equation}
\label{model}
\min_{x'_i, z_i}  \sum_{j=1,2} \sum_{i=1,2} \frac{1}{2} \| \Phi_j(x'_i) - D^j_i z^j_i \|_2^2 + \lambda \| z^j_i \|_1 \,\,\,s.t. ~x=x'_1 + x'_2~.
\end{equation}
Problem (\ref{model}) is minimized with an alternating gradient descent between $x'_i$ and  $z^j_i$. 
Fixing the $z^j_i$ variables and minimizing with respect to $x'_i$ requires locally inverting the scattering 
operators $\Phi_j$, which amounts to solve an overcomplete phase recovery problem and 
can be solved with gradient descent, as shown in \cite{bruna2013audio}. 
Fixed $x'_i$, solving for $z^j_i$ are four independent $\ell_1$ non-negative sparse coding problems, which can be solved 
efficiently. In this work, we use the SPAMS package \cite{mairal2009online}. 
% 
Note that unlike standard NMF problems, the optimization in \ref{model} is carried out directly on the temporal signal.
%We are looking for a decomposition in which the signals are well represented by the models at different
%temporal resolutions.

%Explain the greedy algorithm.
When the analysis operators $\Phi_j$ are able to produce sparse representations of the sources, 
then at each level we have

\begin{eqnarray}
\sum_{i=1,2}  \| \Phi_j(x'_i) - D^j_i z^j_i \|_2^2 \approx \| \Phi_j(x) - D^j_1 z_1 - D^j_2 z_2 \|_2^2 ~,
%&&\| \Phi_j(x'_1) + \Phi_j(x'_2) - \sum_{i=1,2} D^j_i z_i \|_2^2  \approx \| \Phi_j(x) - D^j_1 z_1 - D^j_2 z_2 \|_2^2 ~,
\end{eqnarray}
which can be used in practice to produce a greedy initialization for (\ref{model}) as follows. 
%We first obtain $\widehat{\Phi_2(x_i)}= D^2_i z^{2*}_i$, where the $z^{2*}_i$ are defined as 
%$$z^{2*}_i = \arg\min_{z_i \geq>0} \frac{1}{2}\| \Phi_2(x) - \sum_{i=1,2} D^2_i z_i \|_2^2 +\lambda \| z_i \|_1~.$$
We obtain an estimator $\widehat{\Phi_1(x_i)}= D^1_i z^{1*}_i$, where the $z^{1*}_i$ are defined as 
$$z^{1*}_i = \arg\min_{z_i \geq>0} \frac{1}{2}\| \widehat{\Phi(x)} - \sum_{i=1,2} D^1_i z_i \|_2^2 +\lambda \| z_i \|_1~.$$
We can produce an estimate $\widehat{x}_i$ from  $\widehat{\Phi_1(x_i)}$ by using the complex phases of $W_1 x$.
This can be thought as running standard NMF with CQT features.

%The supervision provides  prior information on the nature of each of the 
%components. However, high-dimensional speech signals have large variability, 
%most of which is uninformative for the purposes of estimating $x_i$ in (\ref{ssep}).
%The training data can be exploited more efficiently in this multi-resolution domain. 
In the proposed framework, the decomposition needs to be coherent at both levels of temporal resolution.
The first level representation is well located temporally and allows for a high resolution rerepentation of the signals,
while the deeper representations can be thought as a regularizer imposing temporal coherence. 
In the deep representation, intra-class variability given by small pitch and timber variations is linearized up to 
 temporal scales $2^J$ without loosing as much discriminative information as 
the spectrogram \cite{deepscatt,pami}.
%
%
%\subsection{Analysis operator}
%
%Common choices are the (non-negative) spectrogram or the the constant $Q-$transform \cite{}. 
%
%Given the observed noisy signal $x$, we denote it's spectrogram as $\bb{\Phi}(x) = |\bb{X}| \in \RR^{m \times n}$ comprising $m$ frequency bins and $n$ temporal frames. $\bb{X} \in \CC^{m \times n}$ contains in it's $i-$th column the Discrete Fourier Transform (DFT) of the $i-th$ frame of $x$, $\bb{x}_i\in \RR^{m}$ ,and it is given by, $\bb{X}_i = \bb{W}^\Tr \bb{x}_i$.
%
%
