\section{Scattering transform}
\label{scattsec}

Discriminative features having longer temporal context can be constructed with the 
scattering transform \cite{deepscatt,pami}. 
While these features have shown excellent performance in various classification tasks, 
in the context of source separation we require a representation that not only captures
long-range temporal structures, but also preserves as much discriminability as possible.
For this reason, we construct a multi-level representation consisting of a pyramid of scattering coefficients with different temporal resolutions at each level. 
This section reviews its definition and main properties when applied to speech signals.

\subsection{Wavelet Filter Bank}

A wavelet $\psi (t)$ is a band-pass filter with good frequency and spatial localization.
 We consider a complex wavelet with a quadrature phase, 
whose Fourier transform satisfies
$\mathcal{F} \psi(\om) \approx 0$ for $\om < 0$.
We assume that the center frequency of $\mathcal{F} \psi$ is $1$ and 
that its bandwidth is of the order of $Q^{-1}$. 
Wavelet filters centered
at the frequencies $\lambda = 2^{j/Q}$ are computed by dilating $\psi$:
$\psi_\la (t) = \la\, \psi(\la\, t)$, and hence $\mathcal{F} \psi_\la (\om) = \widehat \psi(\la^{-1} \omega)$.
We denote by $\Lambda$ the index set of $\la = 2^{j/Q}$ over
the signal frequency support, with $j \leq J$, 
and we impose that these filters fully cover the positive frequencies:
\begin{equation}
\label{consdf}
\forall \om \geq 0~,~ 1- \epsilon 
\leq |\mathcal{F}{\phi}(\omega)|^2 + \frac 1 2 \sum_{\la \in \Lambda} |\mathcal{F}{\psi}_{\lambda}(\omega)|^2 \leq 1~,
\end{equation}
for some $\epsilon <1$, where $\phi(t)$ is the lowpass filter carrying the 
low frequency information at scales larger than $2^J$.
The resulting filter bank has a constant number $Q$ of bands per 
octave. The wavelet transform of a signal $x(t)$ is
\[
W x = \{x \ast \phi(t)~,~x \ast \psi_\la(t)  \}_{\la \in \Lambda}~.
\]
Thanks to (\ref{consdf}), one can verify that 
\begin{equation}
\label{lp_basic}
\| x \|^2 (1- \epsilon) \leq \| x \ast \phi \|^2 + \sum_{\la \in \Lambda} \| x \ast \psi_\la \|^2  \leq \| x \|^2~.
\end{equation}

\subsection{Joint Time-Frequency Pyramid Scattering}

Scattering coefficients provide a nonlinear representation 
computed by iterating over wavelet transforms and complex modulus nonlinearities. 
%We sample the scalograms at critical rate given by . Note that this contrasts with standard scattering transform, were a fixed bandwidth smoothing kernel is used at every layer \cite{deepscatt,pami}.  
%
%First order scattering coefficients
%are local averages of wavelet amplitudes:
%\[
%\forall \la \in \La~~,~~S x(t,\la) = |\, x \ast \psi_\la| \ast \phi(t)~. 
%\]
%The Q-factor $Q_1$ adjusts the frequency resolution of 
%these wavelets, and for speech it is typically around $Q_1=32$.
% Due to the temporal average, first order scattering coefficients 
%provide no information on the time variation of the scalogram
%$|x \ast \psi_{\la_1} (t)|$ at temporal scales smaller than $2^J$. 
%It averages all modulations and 
%transient events, and thus loses perceptually important information.
We start by removing the complex phase of wavelet coefficients in $W^1x$ with a 
complex modulus nonlinearity. We arrange these first layer coefficients as nodes in the first level of the tree,
\[
|W^1| x = \{ x^1_i \}_{i=1\dots 1+|\Lambda|}= \{x \ast \phi_1(\Delta_1 n)~,~|x \ast \psi_{1,\lambda_1}(\Delta_1 n)|  \}_{\lambda \in \Lambda}~.
\]
where $\Delta_1$ is the critical sampling rate of the highest frequency wavelet sub-band 
(the reciprocal of the largest bandwidth present
in the filter bank) and $\psi_1$ has bandwidth $Q_1^{-1}$.
These first layer coefficients give localized information both in
time and frequency, with a trade-off dictated by the Q factor, $Q_1$, that adjusts the frequency resolution of 
these wavelets.  For speech a typical choice is around $Q_1=32$.

In order to increase the robustness of the representation, we transform each of the down sampled signals from this first layer
with a new wavelet filter bank and take the complex modulus of the oscillatory component. 
In order to sample each channel using the same temporal resolution, this time we apply the lowpass anti-aliasing filter to the demodulated channels:
%For simplicity, we assume a dyadic transformation, 
%which reduces the filter bank to a pair of conjugate mirror filters $\{ \phi_2, \psi_2\} $ \cite{wavelettour}, 
%carrying respectively the low-frequencies and high-frequencies of the discrete signal from above the tree:
\begin{equation}
\label{scatnonjoint}
|W^2| x = \{ x^1_i \ast \phi_{2}  (\Delta_2 n)~,~| x^1_i \ast \psi_{2,\lambda_2} | \ast \phi_{2}(\Delta_2 n)| \}_{i=1\dots |W^1|}~,
\end{equation}
where $\Delta_2$ is this time the critical sampling rate of the averaging filter $\phi_2$. 
The multiscale variations of each envelope specify the amplitude modulations of $x(t)$ \cite{deepscatt} and  thus have the capacity to detect rhythmic structures appearing at different frequency bands. The Q-factor $Q_2$ of the second family of wavelets $\psi_{2,\la_2}$ 
controls the time-frequency resolution of the transform. 
Since the envelopes $| x \ast \psi_{\la_1}|$ have bandwidth $\sim 2^{-j} Q_1^{-1}$, 
one typically chooses dyadic $Q_2=1$ second order wavelets.
Scattering coefficients have a negligible amplitude for
$\la_2 > \la_1$ because $|x \ast \psi_{\la_1}|$ is a regular envelop
whose frequency support is below $\la_2$ \cite{pami}. 
Scattering coefficients are thus computed only for $\la_2 < \la_1$. 

%Second order scattering coefficients recover information on
%audio modulations and pitch temporal variations by computing
%the wavelet coefficients of the envelopes $|x \ast \psi_{\la_1}|$, and their
%local averages:
%\[
%\forall \la_2~~,~~
%S x(t,\la_1,\la_2) = ||\, x \ast \psi_{\la_1}| \ast \psi_{\la_2}| \ast \phi(t)~ .
%\]
%These multiscale variations of each envelope $|x \ast \psi_{\la_1}|$ 
%specify the amplitude modulations of $x(t)$ \cite{deepscatt} and 
%thus have the capacity to detect rythmic structures appearing at different 
%frequency bands. 
%The Q-factor $Q_2$ of the second family of wavelets $\psi_{\la_2}$ 
%controls the time-frequency resolution of the transform. 
%Since the envelopes $| x \ast \psi_\la|$ have bandwidth $\sim 2^{-j} Q_1^{-1}$, 
%one typically chooses dyadic $Q_2=1$ second order wavelets.
%Scattering coefficients have a negligible amplitude for
%$\la_2 > \la_1$ because $|x \ast \psi_{\la_1}|$ is a regular envelop
%whose frequency support is below $\la_2$ \cite{pami}. 
%Scattering coefficients are thus computed only for $\la_2 < \la_1$. 

%
%
%Applying more wavelet transform envelopes defines
%scattering coefficients at any order $m \geq 1$:
%\begin{equation}
%\label{expansdf}
%S I(\la_1,...,\la_m; t) =  |~|I \ast \psi_{\la_1}| \ast ... | \ast \psi_{\la_m}| \ast \phi(t)~.
%\end{equation}
%By iterating on the inequality (\ref{lp_basic}), one can 
%verify \cite{stephane}
%that the Euclidean norm of scattering coefficients
%\begin{equation}
%\label{enfsondf}
%\|S I\|^2 = \sum_{m=1}^{\infty} \sum_{(\la_1,...,\la_m) \in \La_m}
%\|S I (\la_1,...,\la_m; \cdot )\|^2 
%\end{equation}
%satisfies
%\[
%\|S I\|^2 \leq  \| I \|^2~.
%\]
%
%For most audio signals, the energy of the scattering
%vector $\|S I\|^2$ is concentrated over first and second layers. 
%In practice, we thus only compute $S I(\la_1)$ and
%$S I(\la_1,\la_2)$ for $1 \leq \la_1 = 2^{j_1/Q_1} \leq N$
%and $1 \leq \la_2 = 2^{j_2/Q_2} < \la_1$. 
%
Scattering transforms have been extended along the frequency
variables to capture the joint time-frequency 
variability of spectral envelopes and therefore provide 
 representations locally stable to pitch variations \cite{deepscatt}. 
We denote $\gamma = \log_2 \lambda_1$, and consider
the scalogram as a two-dimensional function of $\gamma$ and $t$: 
\[
F (\gamma, t) = |x \ast \psi_{2^{\gamma}} (t)|~.
\]

In this work, we consider a second layer scattering with 
a separable wavelet transform $F \ast \overline{\psi}_{\gamma_2, \la_2} (\gamma, t)~,$
with
$$\overline{\psi}_{{\gamma_2}, {\la_2}}(\gamma,t) = \widetilde{\psi}_{\gamma_2}(\gamma) \psi_{2,\la_2}(t)~,~\overline{\psi}_{{0}, {\la_2}}(\gamma,t) = \widetilde{\phi}(\gamma) \psi_{2,\la_2}(t)~$$
$$\overline{\psi}_{{\gamma_2}, {0}}(\gamma,t) = \widetilde{\psi}_{\gamma_2}(\gamma) \phi_{2}(t)~,~\overline{\phi}(\gamma,t) = \widetilde{\phi}(\gamma) \phi_{2}(t)~$$
By replacing $\psi$ and $\phi$ in (\ref{scatnonjoint}) by $\overline{\psi}$ and $\overline{\phi}$ respectively, we obtain the joint scattering pyramid transform.
In this implementation, 
we choose temporal wavelets $\psi(t)$ to be dyadic complex Morlet wavelets,
and $\widetilde{\psi}$ to be dyadic real Haar wavelets to preserve good frequency localization,
with no frequency downsampling. 

%The resulting second order scattering coefficients are thus
%\[
%\tilde{S} x (t, \la_1, \gamma_2,  \la_2) = 
%|F \ast \bar \Psi_{\gamma_2, \la_2} | \ast \overline{\phi}( \log_2 \la_1, t)~,
%\]
%where $\overline{\phi}(\gamma, t)$ is a two-dimensional blurring kernel with temporal 
%scale $2^J$ and log-frequency scale $2^{J_h}$. 
We can reapply the same operator as many times $k$ as desired until reaching a temporal 
context $T = \Delta_k$, but in this implementation we demonstrate the method with $k\leq 2$.
If the wavelet filters are chosen such that they define a non-expansive mapping \cite{pami}, it results that every layer defines a metric which is increasingly contracting:

$$\| |W^k| x - |W^k| x' \| \leq \| |W^{k-1}| x - |W^{k-1}| x' \| \leq \| x - x' \| ~.$$

Every layer thus produces new feature maps at a lower temporal resolution. 
In the end we obtain a tree of different representations,
$$\Phi_j(x) = |W^j| x, \quad \textrm{with} \quad j=1,\ldots, k.$$ 


%\subsection{Pyramid Wavelet Scattering}
%
%Therefor, instead of using a fixed bandwidth smoothing kernel 
%at all layers of the representation we sample each layer at critical rate. In this way, we cunstruct a pyramid of
%scattering coefficients with a different temporal resolutions at each level.
%
%
%
%These first layer coefficients give localized information both in time and frequency, 
%with a trade-off dictated by the $Q$ factor.
%
%$$|W^2| x = \{ x^1_i \ast \phi_2 (2 n)~,~| x^1_i \ast \psi_2 (2n)| \}_{i=1\dots |W^1|}~.$$
%Every layer thus produces new feature maps at a lower temporal resolution.
%
%We can reapply the same operator as many times $k$ as desired until reaching a temporal 
%context $T = 2^k \Delta_1$. If the wavelet filters are chosen such that they define a non-expansive mapping \cite{pami}, 
%it results that every layer defines a metric which is increasingly contracting:
%
%$$\| |W^k| x - |W^k| x' \| \leq \| |W^{k-1}| x - |W^{k-1}| x' \| \leq \| x - x' \| ~.$$

%
%
%\subsection{Analysis operator}
%
%Common choices are the (non-negative) spectrogram or the the constant $Q-$transform \cite{}. 
%
%Given the observed noisy signal $x$, we denote it's spectrogram as $\bb{\Phi}(x) = |\bb{X}| \in \RR^{m \times n}$ comprising $m$ frequency bins and $n$ temporal frames. $\bb{X} \in \CC^{m \times n}$ contains in it's $i-$th column the Discrete Fourier Transform (DFT) of the $i-th$ frame of $x$, $\bb{x}_i\in \RR^{m}$ ,and it is given by, $\bb{X}_i = \bb{W}^\Tr \bb{x}_i$.
%
%
